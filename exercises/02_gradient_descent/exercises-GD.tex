\documentclass{scrartcl}

\usepackage[utf8]{inputenc}
\usepackage[T1]{fontenc}

\usepackage[english]{babel}
\usepackage{amsmath}
\usepackage{cleveref}
\usepackage{amssymb}
\usepackage{mathtools}

%%Numbers, expectation
\newcommand{\N}{\mathbb{N}}
\newcommand{\E}{\mathbb{E}}
\renewcommand{\P}{\mathbb{P}}
\newcommand{\Var}{\mathbb{V}}
\newcommand{\R}{\mathbb{R}}
\newcommand{\D}{\mathcal{D}}
\newcommand{\B}{\mathcal{B}}
\newcommand{\Dh}{\D_h}
\renewcommand{\phi}{\varphi}
\newcommand*\diff{\mathop{}\!\mathrm{d}} % integral

%% mathoperator
\DeclareMathOperator*{\argmax}{arg\,max}
\DeclareMathOperator*{\argmin}{arg\,min}
\DeclareMathOperator*{\dom}{dom}
\DeclareMathOperator*{\sign}{sign}
\DeclareMathOperator*{\diag}{diag}

\DeclareMathOperator*{\Cov}{Cov}
\DeclareMathOperator*{\Cor}{Corr}

%proximal operator
\newcommand{\prox}[3][]{\operatorname{prox}^{#1}_{#2}\left(#3 \right)}

\usepackage{xcolor}

%% sort citations by increasing number
\usepackage[sort,nocompress]{cite}

\usepackage{graphicx}% http://ctan.org/pkg/graphicx
\graphicspath{{../figures/}{../../figures}} %Setting the graphicspath
\usepackage{caption,subcaption}

\usepackage{tikz}
\usepackage{pgfplots}
\usetikzlibrary{backgrounds}
\usetikzlibrary{intersections}
\usepgfplotslibrary{fillbetween}

% \usepackage[right]{showlabels}


%%
\theoremstyle{plain}
\newtheorem{prop}{Proposition}[section]
\newtheorem{algo}{Algorithm}[section]
\newtheorem{assumption}{Assumption}
\theoremstyle{remark}
\newtheorem{remark}{Remark}[section]

% cref
\crefname{assumption}{Assumption}{Assumptions}
\crefname{equation}{}{}

\usepackage{autonum}

\usepackage{bm} %% bold math symbols

\usepackage{bbm} %% for \mathbbm{1}


% algorithmic environment
\usepackage{algorithm}
\usepackage[noend]{algpseudocode}

% for some reason this was required on one void linux installation (but not the other)
\usepackage{sansmathaccent}
\pdfmapfile{+sansmathaccent.map}

\author{Axel Böhm}

% shows which section we're in
\usetheme{Darmstadt}

% page number
\setbeamertemplate{footline}[frame number]
\setbeamercolor{page number in head/foot}{fg=gray}


% display things like onslide or visible already before but grayed out
\setbeamercovered{transparent}

\theoremstyle{definition}
\newtheorem{definition}{Definition}[section]
\newtheorem{lemma}{Lemma}[section]
\usepackage{enumerate}

\begin{document}


\section*{Problem set: Gradient descent}%


\paragraph{Exercise (i)} (7P) Enter the missing code snippets in the jupyter notebook. Partial credit will be awarded.


\begin{definition}
  For a matrix $Q \in \R^{m \times d}$ its \textbf{operator norm} is defined as
  \begin{equation}
    \Vert Q \Vert_{op} := \sup_{x \in \R^d: \Vert x \Vert \le 1} \Vert Q x \Vert,
  \end{equation}
  but for convenience we will mostly write $\Vert Q \Vert$ (so no subscript) but mean the operator norm.
\end{definition}

\begin{lemma}%
  The operator norm of a matrix $Q$ fulfills
  \begin{equation}
    \Vert Qx \Vert \le \Vert Q \Vert_{op} \Vert x \Vert.
  \end{equation}
  In particular, every linear map (given by a matrix $Q$) is Lipschitz continuous with Lipschitz constant $\Vert Q \Vert$.
\end{lemma}


\paragraph{Exercise (ii)} (2P) Prove that the quadratic function
\begin{equation}
  f(x) = \frac{1}{2} x^T Q x +b^T x + c
\end{equation}
is \textbf{smooth} with parameter $\Vert Q \Vert$.


\paragraph{Exercise (iii)} (1P) Suppose that we have observations $(x_i, y_i)$ which are \textbf{centered}, meaning that $\sum_{i=1}^{n}x_i = 0 = \sum_{i=1}^{n}y_i$. Let $(b^*, w^*)$ be the global minimum of the least squares objective
\begin{equation}
  f(b, w) = \sum_{i=1}^{n} {(b + w^T x_i - y_i)}^2.
\end{equation}
Prove that $b^*=0$.



\end{document}
