\documentclass{scrartcl}

\usepackage[utf8]{inputenc}
\usepackage[T1]{fontenc}

\usepackage[english]{babel}
\usepackage{amsmath}
\usepackage{cleveref}
\usepackage{amssymb}
\usepackage{mathtools}

%%Numbers, expectation
\newcommand{\N}{\mathbb{N}}
\newcommand{\E}{\mathbb{E}}
\renewcommand{\P}{\mathbb{P}}
\newcommand{\Var}{\mathbb{V}}
\newcommand{\R}{\mathbb{R}}
\newcommand{\D}{\mathcal{D}}
\newcommand{\B}{\mathcal{B}}
\newcommand{\Dh}{\D_h}
\renewcommand{\phi}{\varphi}
\newcommand*\diff{\mathop{}\!\mathrm{d}} % integral

%% mathoperator
\DeclareMathOperator*{\argmax}{arg\,max}
\DeclareMathOperator*{\argmin}{arg\,min}
\DeclareMathOperator*{\dom}{dom}
\DeclareMathOperator*{\sign}{sign}
\DeclareMathOperator*{\diag}{diag}

\DeclareMathOperator*{\Cov}{Cov}
\DeclareMathOperator*{\Cor}{Corr}
\DeclareMathOperator*{\Id}{Id}

%proximal operator
\newcommand{\prox}[3][]{\operatorname{prox}^{#1}_{#2}\left(#3 \right)}

\usepackage{xcolor}

%% sort citations by increasing number
\usepackage[sort,nocompress]{cite}

\usepackage{graphicx}% http://ctan.org/pkg/graphicx
\graphicspath{{../figures/}{../../figures}{../../memes}} %Setting the graphicspath
\usepackage{caption,subcaption}

\usepackage{tikz}
\usepackage{pgfplots}
\usetikzlibrary{backgrounds}
\usetikzlibrary{intersections}
\usepgfplotslibrary{fillbetween}

% \usepackage[right]{showlabels}


%%
\theoremstyle{plain}
\newtheorem{prop}{Proposition}[section]
\newtheorem{algo}{Algorithm}[section]
\newtheorem{assumption}{Assumption}
\theoremstyle{remark}
\newtheorem{remark}{Remark}[section]

% cref
\crefname{assumption}{Assumption}{Assumptions}
\crefname{equation}{}{}

\usepackage{autonum}

\usepackage{bm} %% bold math symbols

\usepackage{bbm} %% for \mathbbm{1}


% algorithmic environment
\usepackage{algorithm}
\usepackage[noend]{algpseudocode}

% for some reason this was required on one void linux installation (but not the other)
\usepackage{sansmathaccent}
\pdfmapfile{+sansmathaccent.map}

\author{Axel Böhm}

% shows which section we're in
\usetheme{Darmstadt}

% page number
\setbeamertemplate{footline}[frame number]
\setbeamercolor{page number in head/foot}{fg=gray}


% display things like onslide or visible already before but grayed out
\setbeamercovered{transparent}

% set the itemize item symbol as a diamond
\setbeamertemplate{itemize item}{$\diamond$}
% set the itemize subitem symbol as a triangle
\setbeamertemplate{itemize subitem}{$\blacktriangleright$}

% set the enumerate item symbol as a roman numbers
\setbeamertemplate{enumerate item}{(\roman{enumi})}

\theoremstyle{definition}
\newtheorem{definition}{Definition}[section]
\newtheorem{lemma}{Lemma}[section]
\usepackage{enumerate}

\begin{document}


\section*{Problem set: Gradient descent and Fixed points}%



\paragraph{Exercise (i)} (1P)  Let $f_1, f_2, \dots, f_m$ be smooth with parameters $L_1, L_2, \dots L_m$. Show that the function $f:= \sum_{i=1}^{m}f_i$ is smooth with parameter $\sum_{i=1}^{m}L_i$.

\paragraph{Exercise (ii)} (1P)  Let $f$ be smooth with parameter $L$ and $A$ a matrix. Show that $f\circ A$ is smooth with parameter $L \Vert A \Vert^2$.




\subsection*{Computing Fixed Points}%

Gradient descent turns up in a surprising number of situations which apriori have nothing to do with optimization.
In this exercise we will see how computing the fixed point of functions can be seen as a form of gradient descent.
Suppose that we have a $1$-Lipschitz continuous function $g : \R \to \R$ such that we want to solve for
\begin{equation}
  g(x) = x .
\end{equation}
A simple strategy for finding such a fixed point is to run the following algorithm: starting from an arbitary $x_0$,
we iteratively set
\begin{equation}
  \label{fpi}
  x_{k+1} = g(x_k) .
\end{equation}

\paragraph{Exercise (iii)} (3P) Enter the missing code snippets in the jupyter notebook. Partial credit will be awarded.

We will try solve for $x$ starting from $x_0 = 1$ in the following two equations:
\begin{equation}
  \label{log}
  x = \log(1 + x), \quad \text{and} \quad x = \log(2 + x).
\end{equation}
What difference do you observe in the rate of convergence between the two problems? Let’s understand why this happens:

\paragraph{Exercise (iv)} (3P) Theoretical fixed point questions.
\begin{itemize}
  \item We want to re-write the update~\eqref{fpi} as a step of gradient descent. To do this, we need to find a function $f$
        such that the gradient descent update is identical to~\eqref{fpi}:
        \begin{equation}
          x_{k+1} = x_k - \alpha f'(x_k) = g(x_k) .
        \end{equation}
        Derive such a function $f$.
  \item Give sufficient conditions on $g$ to ensure convergence of procedure~\eqref{fpi}. What $\alpha$ would you need to pick?
        Hint: We know that gradient descent on f with fixed step-size converges if $f$ is convex and smooth. What
        does this mean in terms of $g$?
  \item What condition does $g$ need to satisfy to ensure linear convergence? Are these satisfied for the problems in~\eqref{log}
\end{itemize}


\end{document}
