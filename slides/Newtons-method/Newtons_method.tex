\documentclass{beamer}
\usepackage[utf8]{inputenc}

\usepackage[utf8]{inputenc}
\usepackage[T1]{fontenc}

\usepackage[english]{babel}
\usepackage{amsmath}
\usepackage{cleveref}
\usepackage{amssymb}
\usepackage{mathtools}

%%Numbers, expectation
\newcommand{\N}{\mathbb{N}}
\newcommand{\E}{\mathbb{E}}
\renewcommand{\P}{\mathbb{P}}
\newcommand{\Var}{\mathbb{V}}
\newcommand{\R}{\mathbb{R}}
\newcommand{\D}{\mathcal{D}}
\newcommand{\B}{\mathcal{B}}
\newcommand{\Dh}{\D_h}
\renewcommand{\phi}{\varphi}
\newcommand*\diff{\mathop{}\!\mathrm{d}} % integral

%% mathoperator
\DeclareMathOperator*{\argmax}{arg\,max}
\DeclareMathOperator*{\argmin}{arg\,min}
\DeclareMathOperator*{\dom}{dom}
\DeclareMathOperator*{\sign}{sign}
\DeclareMathOperator*{\diag}{diag}

\DeclareMathOperator*{\Cov}{Cov}
\DeclareMathOperator*{\Cor}{Corr}

%proximal operator
\newcommand{\prox}[3][]{\operatorname{prox}^{#1}_{#2}\left(#3 \right)}

\usepackage{xcolor}

%% sort citations by increasing number
\usepackage[sort,nocompress]{cite}

\usepackage{graphicx}% http://ctan.org/pkg/graphicx
\graphicspath{{../figures/}{../../figures}} %Setting the graphicspath
\usepackage{caption,subcaption}

\usepackage{tikz}
\usepackage{pgfplots}
\usetikzlibrary{backgrounds}
\usetikzlibrary{intersections}
\usepgfplotslibrary{fillbetween}

% \usepackage[right]{showlabels}


%%
\theoremstyle{plain}
\newtheorem{prop}{Proposition}[section]
\newtheorem{algo}{Algorithm}[section]
\newtheorem{assumption}{Assumption}
\theoremstyle{remark}
\newtheorem{remark}{Remark}[section]

% cref
\crefname{assumption}{Assumption}{Assumptions}
\crefname{equation}{}{}

\usepackage{autonum}

\usepackage{bm} %% bold math symbols

\usepackage{bbm} %% for \mathbbm{1}


% algorithmic environment
\usepackage{algorithm}
\usepackage[noend]{algpseudocode}

% for some reason this was required on one void linux installation (but not the other)
\usepackage{sansmathaccent}
\pdfmapfile{+sansmathaccent.map}

\author{Axel Böhm}

% shows which section we're in
\usetheme{Darmstadt}

% page number
\setbeamertemplate{footline}[frame number]
\setbeamercolor{page number in head/foot}{fg=gray}


% display things like onslide or visible already before but grayed out
\setbeamercovered{transparent}


\title{Newton's and Quasi-Newton Methods}
\date{\today}


\begin{document}
\maketitle
\frame{\tableofcontents[currentsection]}

\section{Introduction}

\begin{frame}
  \frametitle{$1$-dimensional case: Newton-Raphson method}
  \begin{minipage}{0.48\textwidth}
    \textcolor{blue}{Objective:} Find zero of differentiable $f: \R \to \R$.\\
    \textcolor{blue}{Strategy:} Solve
    \begin{equation}
      f(x_k) + f'(x_k) (x - x_k) = 0.
    \end{equation}
    \textcolor{blue}{Method:} Gives
    \begin{equation}
      x_{k+1} = x_k - \frac{f(x_k)}{f'(x_k)}
    \end{equation}
  \end{minipage}
  \hfill
  \begin{minipage}{0.48\textwidth}
    \begin{figure}[ht]
      \centering
      \includegraphics[width=\textwidth,height=\textheight,keepaspectratio]{newton-raphson}
      % \caption{\label{fig:label} }
    \end{figure}
  \end{minipage}
\end{frame}

\begin{frame}
  \frametitle{The Babylonian method}
  \begin{itemize}
    \item \textcolor{blue}{compute square root} of $R \in \R_+$
    \item find zero of $f(x) = x^2 - R$
    \item use Newton-Raphson:
          \begin{equation}
            x_{k+1} = x_k - \frac{f(x_k)}{f'(x_k)} = x_k - \frac{x_k^2 - R}{2 x_k} = \frac12 \left( x_k + \frac{R}{x_k} \right)
          \end{equation}
    \item Starting from $x_0 > 0$ we have
          \begin{equation}
            x_{k+1} = \frac12 \left( x_k + \frac{R}{x_k} \right) \ge \frac{x_k}{2}.
          \end{equation}
    \item Starting from $x_0 = R \ge 1$, it takes $\mathcal{O}(\log R)$ steps to get to $x_k - \sqrt{R} < \frac12$.
  \end{itemize}
\end{frame}

\begin{frame}
  \frametitle{The Babylonian method - Takeoff}
  Note that
  \begin{equation}
    x_{k+1} - \sqrt{R} = \frac12 \left( x_k + \frac{R}{x_k} \right) - \sqrt{R} = \frac{x_k}{2} + \frac{R}{2 x_k} - \sqrt{R} = \frac{1}{2 x_k} {\left( x_k - \sqrt{R} \right)}^2
  \end{equation}
  For simplicity $R \ge 1/4$, then $x_k \ge \sqrt{R} \ge 1/2$. Hence
  \begin{equation}
    x_{k+1} - \sqrt{R} = \frac{1}{2 x_k} {\left( x_k - \sqrt{R} \right)}^2 \le {\left( x_k - \sqrt{R} \right)}^2
  \end{equation}
  If $x_0 - \sqrt{R} < \frac12$ (ensured after $\mathcal{O}(\log R)$ steps).
  \begin{equation}
    x_{k} - \sqrt{R} \le {\left( x_0 - \sqrt{R} \right)}^{2^k} \le {\left(\frac12\right)}^{2^k}
  \end{equation}
  To achieve $x_k - \sqrt{R} < \epsilon$ we only need $k = \log \log (\epsilon^{-1})$ steps!
\end{frame}

\begin{frame}
  \frametitle{The Babylonian method - Example}
  $R=1000$, in double arithmetic
  \begin{itemize}
    \item $7$ steps to get to $x_7 - \sqrt{1000} < 1/2$
          \item $3$ steps to get to $\sqrt{1000}$ up to \textit{machine precision}
          \item First phase: $\approx$ \textcolor{blue}{one more correct digit} per iteration
          \item Second phase: $\approx$ \textcolor{blue}{double the number of correct digits} per iteration
  \end{itemize}

  \begin{center}
    In practice: $\log \log x \le 5$.
  \end{center}

\end{frame}

\section{Newton's method}%
\label{sec:}

\begin{frame}
  \frametitle{Newton's method for optimization}
  \textcolor{blue}{Goal:} Find global minimum $x^*$ of convex, differentiable function $f$.
  \textcolor{blue}{Strategy:} Search for zero of derivative.

  \textbf{$1$-dimensional case:} Apply Newton-Raphson method to $f'$:
  \begin{equation}
    x_{k+1} = x_k - \frac{f'(x_k)}{f''(x_k)} = x_k - {f''(x_k)}^{-1} f(x_k)
  \end{equation}
  (requires \textcolor{blue}{twice} differentiable and $f'' > 0$)

  \textbf{$d$-dimensional case:} Newtons methods for minimizing convex $f: \R^d \to \R$:
  \begin{equation}
    x_{k+1} = x_k - \nabla^2 f(x_k) \nabla f(x_k)
  \end{equation}

\end{frame}


\begin{frame}
  \frametitle{Newton's method as adaptive gradient descent}
  General update scheme:
  \begin{equation}
    x_{k+1} = x_k - H(x_k) \nabla f(x_k)
  \end{equation}
  for some matrix $H(x) \in \R^{d \times d}$.
  \begin{itemize}
    \item \textcolor{blue}{Newton's method}: $H = {\nabla^2 f(x_k)}^{-1}$.
    \item \textcolor{blue}{Gradient descent}: $H = \alpha \Id$
  \end{itemize}
  \vspace{1cm}
  Newton's methods \textbf{adapts} to the local geometry of $f$ at $x_k$ \\
  $\rightarrow$ \textit{no need for choosing a stepsize}.
\end{frame}

\begin{frame}
  \frametitle{Convergence in one step on quadratic functions}
  A \textcolor{blue}{quadratic} function
  \begin{equation}
    f(x) = \frac12 x^T M x + q^T x + c
  \end{equation}
  is called \textit{nondegenerate} if $M \in \R^{d \times d}$ is invertible.
  \begin{itemize}
    \item $x^* := M^{-1}q$ is the unique solution of $\nabla f(x) = 0$
    \item $x^*$ is the unique global minimum if $f$ is convex
  \end{itemize}
  \begin{lemma}%
    On nondegenerate quadratic functions with arbitrary starting point $x_0$, Newtons method yields $x_1=x^*$
  \end{lemma}
  \begin{proof}
    We have $\nabla f(x) = Mx -q$ and $\nabla^2 f(x) = M$. Therefore
    \begin{equation}
      x_1 = x_0 - \nabla^2 f(x_0) \nabla f(x_0) = x_0 - M^{-1}(M x_0 - q) = M^{-1}q = x^*.
    \end{equation}
  \end{proof}
\end{frame}

\begin{frame}
  \frametitle{Affine Invariance}
  Newton’s method is \textcolor{blue}{affine invariant}
  (invariant under any invertible affine transformation):
  Denote the Newton step for $h$ by
  \begin{equation}
    N_h(x) := x - {\nabla^2 h }^{-1} \nabla h(x).
  \end{equation}
  \begin{lemma}%
    Let $f: \R^d \to \R$ be twice differentiable, $A \in \R^{d \times d}$ an invertible matrix and $b \in \R^d$.
    \begin{equation}
      g(x) = Ax + b.
    \end{equation}
    Then
    \begin{equation}
      N_{f \circ g} = g^{-1} \circ N_f \circ g
    \end{equation}
  \end{lemma}
\end{frame}

\begin{frame}
  \frametitle{}
  \begin{figure}[ht]
    \centering
    \includegraphics[width=\textwidth,height=\textheight,keepaspectratio]{newton_aff_invariance}
    % \caption{\label{fig:label} }
  \end{figure}
  Gradient descent suffers if coordinates are at different scales; Newton's method doesn't.
\end{frame}


\begin{frame}
  \frametitle{Minimizing the second-order Taylor approximation}
  Alternative interpretation of Newton's method:
  Minimize (local) \textcolor{blue}{quadratic model} of $f$.
  \begin{lemma}%
    Let $f$ be conve, twice differentiable and $\nabla^2 f(x) \succ 0$. Then $x_{k+1}$ resulting from \textbf{Newton's step} satisfies
    \begin{equation}
      x_{k+1} = \argmin_{x \in \R^d} f(x_k) + \langle \nabla f(x_k), x-x_k \rangle + \frac12 \langle x-x_k, \nabla^2 f(x_k) (x-x_k)  \rangle
    \end{equation}
  \end{lemma}
\end{frame}


\begin{frame}
  \frametitle{Local Convergence}
  We will prove:\\
  Under suitable conditions on $f$ and \textcolor{blue}{close to the minimum} Newton's method approximates solution up to an error $\epsilon$ in \textcolor{blue}{$\log \log (1/\epsilon)$} iterations.
  \begin{itemize}
    \item much faster than anything so far..
    \item only locally
  \end{itemize}
  We call this a \textcolor{blue}{local convergence} result.\\

  \textcolor{blue}{Global convergence} statements are more difficult to obtain (some only recently).
\end{frame}

\section{Convergence analysis}%
\label{sec:}

\begin{frame}
  \frametitle{Theorem + Technical conditions }
  \begin{theorem}
    Let $f$ be convex with unique global minimum $x^*$, and $X$ a ball around $x^*$ s.t.
    \begin{enumerate}
      \item \textcolor{blue}{Bounded inverse Hessians:} There exists $\mu > 0$
            \begin{equation}
              \Vert {\nabla^2 f(x)}^{-1} \Vert \le \frac{1}{\mu}, \quad \forall x \in X
            \end{equation}
      \item \textcolor{blue}{Lipschitz continuous Hessians:} There exists $B>0$
            \begin{equation}
              \Vert \nabla^2 f(x) - \nabla^2 f(y) \Vert \le B \Vert x-y \Vert, \quad \forall x,y \in X
            \end{equation}
    \end{enumerate}
    Then, for $x_{k+1} = N_f(x_k)$ we have
    \begin{equation}
      \Vert x_{k+1} -x^* \Vert \le \frac{B}{2 \mu} \Vert x_k - x^* \Vert^2
    \end{equation}
  \end{theorem}
\end{frame}


\begin{frame}
  \frametitle{Super-exponential speed}
  \begin{corollary}%
    In the setting of previous theorem, if
    \begin{equation}
      \Vert x_k - x^* \Vert \le \frac{\mu}{B},
    \end{equation}
    then
    \begin{equation}
      \Vert x_k -x^*  \Vert \le \frac{\mu}{B} {\left( \frac{1}{2} \right)}^{2^k-1}
    \end{equation}
  \end{corollary}
  Close to the global minimum, we will reach distance to the minumum less than $\epsilon$ in at most $\log \log (1/\epsilon)$ steps.

  As for the last phase of Babylonian method.
\end{frame}


\begin{frame}
  \frametitle{Super-exponential speed - intuition}

  \begin{itemize}
    \item Almost constant Hessians close to optimality...
    \item so $f$ behaves almost like a quadratic
    \item on which Newton's converge in one step
  \end{itemize}

  \begin{lemma}%
    If
    \begin{equation}
      \Vert x_0 - x^* \Vert \le \frac{\mu}{B}
    \end{equation}
    the Hessians in Newton's method satisfy the \textcolor{blue}{relative error bound}
    \begin{equation}
      \frac{\Vert  \nabla^2 f(x_k) - \nabla^2 f(x^*) \Vert}{\Vert \nabla^2 f(x^*) \Vert} \le {\left( \frac12 \right)}^{2^k-1}.
    \end{equation}
  \end{lemma}
\end{frame}


\begin{frame}
  \frametitle{Proof of convergence theorem}
  We abbreviate $H = \nabla^2 f(x_k)$, $x=x_k$, $x^+ = x_{k+1}$
  \begin{equation}
    \begin{aligned}
      x^+ - x^* &= x - x^* - H^{-1} \nabla f(x) \\
      &= x - x^* + H^{-1}( \nabla f(x^*) - \nabla f(x)) \\
      &= x - x^* + H^{-1} \int_{0}^{1} H(x + t(x^* - x))(x^* - x) \diff t,
    \end{aligned}
  \end{equation}
  \onslide<2->{%
    where we used the fundamental theorem of calculus
    \begin{equation}
      \int_{a}^{b} h' (t) \diff t
    \end{equation}
    with
    \begin{align}
      h(t)  &= \nabla f(x + t(x^* -x )) \\
      h'(t) &= \nabla^2f (x + t(x^* -x ))(x^*-x).
    \end{align}
  }
\end{frame}


\begin{frame}
  \frametitle{Downside of Newton's method}
  \textcolor{blue}{Computational bottleneck} in every step:
  \begin{itemize}
    \item compute Hessian
    \item invert Hessian or solve $\nabla^2 f(x_k) \Delta x = - \nabla f(x_k)$
  \end{itemize}
  Matrix has size $d\times d$, taking $\mathcal{O}(d^3)$ to invert.\\
  In many applications the dimension $d$ is large (too large to even store Hessian).

\end{frame}


\begin{frame}
  \frametitle{The secant method}
  Another iterative method for finding zeros in $1$-d.
  Recall Newton-Raphson:
  \begin{equation}
    x_{k+1} = x_k - \frac{f(x_k)}{f'(x_k)}
  \end{equation}
  Use \textcolor{blue}{finite difference approximation} of $f'(x_k)$:
  \begin{equation}
    f'(x_k) \approx \frac{f(x_k) - f(x_{k-1})}{x_k - x_{k-1}}.
  \end{equation}
  We obtain the \textcolor{blue}{\textbf{secant method}}:
  \begin{equation}
    x_{k+1} = x_k - f(x_k) \frac{f(x_k) - f(x_{k-1})}{x_k - x_{k-1}}.
  \end{equation}

\end{frame}


\begin{frame}
  \frametitle{The secant method}
  \begin{figure}[ht]
    \centering
    \includegraphics[width=0.7\textwidth,keepaspectratio]{secant_method}
  \end{figure}
  Constructs the line through $(x_{k-1}, f(x_{k-1}))$ and $(x_k, f(x_k))$
\end{frame}

\begin{frame}
  \frametitle{}

\end{frame}

\end{document}
